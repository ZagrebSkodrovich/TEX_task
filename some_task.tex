\documentclass[a4paper]{article}
\usepackage[utf8]{inputenc}
\usepackage[english, russian]{babel}
\usepackage{geometry,
    graphicx,
    xcolor,
    listings,
    hyperref,
    minted,
    amsfonts}
\graphicspath{{./images/}}
\geometry{left=20mm,
        right=10mm,
        top=20mm,
        bottom=20mm, }
\setlength{\parindent}{0em}

\title{Домашнее задание в \LaTeX.}
\author{Исаев Георгий, группа 215}
\date{\today}

\begin{document}

\maketitle

\pagenumbering{roman}
\pagenumbering{arabic}

\paragraph{Условие}
По круглому тонкому проводнику радиусом $r$ течет ток силы $I$. 
Найти магнитное поле в центре.

\paragraph{Решение}
Воспользуемся законом Био-Савара-Лапласа:

\begin{equation}
    d\vec{B} = \frac{\mu_0}{4\pi} \cdot \frac{I\left[d\vec{l} \cdot \vec{r}\right]}{r^3}
\end{equation}

Из этого закона следует, что направление вектора $d\vec{B}$ совпадает с направлением
векторного произведения $\left[d\vec{l}, \vec{r}\right]$. Такое же направление
дает и правило правого винта.

Учитывая, что
\[
    \left| \left[d\vec{l}, \vec{r}\right] \right| = dl r \sin(d\vec{l},\vec{r}) = dlr\sin\alpha,
\]

\[
    \mathrm{d}B = \frac{\mu_0}{4\pi} \cdot \frac{I\mathrm{d}l \cdot \sin\alpha}{r^2},
\]

Получим
\[
    B = \int_0^l \frac{\mu_0}{4\pi} \cdot \frac{I\mathrm{d}l \cdot \sin\alpha}{r^2} = 
    \frac{\mu_0I}{4\pi r^2} \int_0^l \mathrm{d}l = \frac{\mu_0I}{4\pi r^2} \cdot 2\pi r = 
    \frac{\mu_0I}{2r}
\]

\end{document}
